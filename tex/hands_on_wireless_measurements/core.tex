\section{Demodulating a capture with GNU Radio}

The teaching team recorded a capture containing five packets sent with the S2LP radio.
As a first introduction to GNU Radio, you will demodulate this capture using the 2-FSK demodulator and CFO estimators you wrote in the previous hands-on.

\subsection{Installing the \texttt{gr-fsk} module}

For this session, we provide you a version of the project that includes both a full software implementation for the MCU and a GNU Radio
module compatible with the S2LP radio. Download this project on Moodle and unzip it.

GNU Radio provides many built-in digital signal processing blocks (similarly to LabView), however these blocks are not sufficient to implement a functional FSK receiver.
The teaching team hence developed a new \href{https://wiki.gnuradio.org/index.php/OutOfTreeModules}{out-of-tree module} named
\texttt{gr-fsk} that contains all the blocks (preamble detection, synchronization, demodulation, packet parser, ...) needed to build an entire receiver chain for the S2LP radio.
Before running GNU Radio, this module must be compiled and installed. To this end, open a terminal, go into the \texttt{telecom/hands\_on\_wireless\_measurements/gr-fsk} folder and execute the following commands:



\begin{bash}
    mkdir build
    cd build/
    cmake ..
    sudo make install
    sudo ldconfig
\end{bash}
The module is now installed.

To ensure GNU Radio will be able to locate the installation directory of this custom package, we need to check the \texttt{PYTHONPATH} environment variable of GNU Radio. In the prompted log of the above installation commands, you should see multiple lines such as \texttt{/usr/local/XXX/dist-packages/fsk/XXX}. Now run the command :


\begin{bash}
echo $PYTHONPATH
\end{bash}

If the path from the compilation log (up to \texttt{dist-packages}) does not appear, we will need to add it to the \texttt{PYTHONPATH} variable with the following command you should adapt to your case :


\begin{bash}
echo "export PYTHONPATH='/usr/local/XXX/dist-packages/:$PYTHONPATH'" >> ~/.bashrc
\end{bash}

\subsection{Running the GNU Radio Companion}

GNU Radio may either be used directly using Python scripts or through a GUI software named GNU Radio Companion. It is much more convenient to use the latter to design and experiment receiver chains.
Start the GNU Radio Companion (\textit{``Programming -> GNU Radio Companion''}) and open the application \texttt{gr-fsk/apps/decode\_capture.grc}.
This application reads a prerecorded capture from a file
and demodulates it using the FSK receiver architecture that has been explained in the simulation hands-on. The operations carried out by the different blocks are exactly identical to those of the simulation framework we provided you.

\begin{bclogo}[couleur = gray!20, arrondi = 0.2, logo=\bcinfo]{GNU Radio parameter adjustement}
    In the GNU Radio GUI, the low-pass filter might be highlighted in red due to an error depending on the Ubuntu release you are using. Double-click on it and on the part highlighted in red and select the corresponding one. (Normally the error, if present, is on the \textit{Window} parameter which should be set to \textit{Hamming})
\end{bclogo}

\texttt{.grc} files are designs which can be graphically edited in the companion, but they cannot directly be executed.
Instead, the companion generates Python scripts based on the \texttt{.grc} design. To generate the corresponding script, press the second button starting from the left (i.e., the one with a box and an arrow) shown in Fig.~\ref{fig:buttons}.
The generated script can then be executed by pressing the \textit{Run} button (third button starting from the left).
\begin{figure}[H]
    \centering
    \includegraphics[scale=1]{figures/buttons.PNG}
    \caption{GNU Radio buttons to generate the Python script, running and stopping the application.}
    \label{fig:buttons}
\end{figure}

\begin{bclogo}[couleur = gray!20, arrondi = 0.2, logo=\bcinfo]{Setting the path of the \textit{File Source}}
    You may need to fix the path of capture in the block \textit{File Source}.
    The capture is located at \texttt{gr-fsk/misc/fsk\_capture.mat}.
    After modifying the design in the companion, re-generate the corresponding Python script before launching the application.
\end{bclogo}

When running the application, the console on the bottom left corner on the screen indicates the events processed by the different blocks.
Since both the estimation of the CFO and the demodulation functions are not implemented in this version of the project,
you should observe that 5 packets are detected with a CFO of \SI{0}{\hertz} and are incorrectly demodulated (all payload bytes are demodulated to zeroes).

\subsection{Modifying the \texttt{gr-fsk} module}

To correctly demodulate the capture, you now need to plug in the demodulator and CFO estimator from the previous hands-on into the \texttt{gr-fsk} module.
The \texttt{gr-fsk} blocks are written in Python and are all located in the folder \texttt{gr-fsk/python}.
Open the file \texttt{gr-fsk/python/demodulation.py}, which implements the \textit{Demodulation} block. For all \texttt{gr-fsk} blocks, the teaching team wrote the boilerplate code that
handles the buffers inside GNU Radio. On the contrary, the operations that are specific to the signal processing (e.g., demodulation, estimating the CFO or STO, ...) have been outsourced
to external functions. These functions have almost similar prototypes to the corresponding functions of the simulation framework.
\textbf{You may hence validate your signal processing functions in the simulation framework before inserting them in GNU Radio.}

\begin{bclogo}[couleur = gray!20, arrondi = 0.2, logo=\bcinfo]{Remark on writing code that is \textit{fast enough}}
    In the simulation framework, your code did not have to be fast to work.
    \textbf{For the measurements}, however, the code should run in \textit{real-time},
    meaning that, if your code is too slow, the packet will arrive faster
    than they are decoded, eventually leading to errors.

    If your code did work in the simulation tests, but not in GNU Radio, please
    be careful to avoid too many for-loops, and prefer using array operations
    directly.

    You can use the \texttt{timeit} Python decorator from
    \texttt{gr-fsk/python/utils} to measure how much time each function takes,
    and identify the ones that are important to optimize.
\end{bclogo}

Nonetheless, it is important that you gain some understanding of how custom GNU Radio blocks are written.
Please take some time to dive into \texttt{demodulation.py}. The functions \texttt{\_\_init\_\_} (initialization of the block with the specified parameters),
\texttt{forecast} (indicates to GNU Radio how many input samples are needed to provide $N$ output samples) and \texttt{general\_work} (actual function that performs the processing) are common to all GNU Radio blocks.
The companion however does not directly read the Python files to understand how a block is implemented. Instead, each block comes with a
YAML file, read by the companion, which includes all the information required by the tool: types of the input and output signals, parameters,
callback functions, ... Open the file \texttt{gr-fsk/grc/fsk\_demodulation.block.yml} and browse it also.

Once you went over the Python and YAML files, put the demodulation function you wrote for the previous hands-on in the external \texttt{demodulate} function. Repeat the same procedure for the CFO estimation in the function
\texttt{cfo\_estimation} of \texttt{gr-fsk/python/synchronization.py}. Then, before running the GNU Radio application again, you need to re-do an installation to propagate your changes:

\begin{bash}
    cd build/
    sudo make install
\end{bash}

\begin{bclogo}[couleur = gray!20, arrondi = 0.2, logo=\bcinfo]{Usage of the build folder}
    Creating and populating the build directory with \texttt{cmake} needs only to be done once.
    Afterwards it is sufficient to only do an install to propagate your changes in the Python files to the system.
\end{bclogo}

After the installation of your modifications, re-launch the application.
If the two functions \texttt{demodulate} and \texttt{cfo\_estimation} are correctly implemented, you should now observe that all 5 packets are rightly demodulated (with the payload bytes increasing succesively from 0 to 99)
and that the CFO of each packet is approximately located around \SI{7800}{\hertz}.

\section{Live demodulation with a cable}

Now that the \texttt{gr-fsk} blocks are fully functional, the next step consists in receiving and demodulating in real-time packets from the MCU.
The project contains a full MCU software implementation with a driver to interact with the S2LP radio.
This software features an evaluation mode of the radio that transmits several packets in a row with different transmit (Tx) output power levels.
To verify the entire chain (Tx and Rx) at a very high signal-to-noise ratio (SNR), we first use an SMA cable to connect the S2LP radio and the LimeSDR Mini.
The reference datasheet of the radio is available at the following link: \href{https://www.st.com/resource/en/datasheet/s2-lp.pdf}{\textcolor{blue}{[S2-LP datasheet]}}.

\subsection{Preparing the setup}

\begin{enumerate}
    \item Connect the S2LP radio on the MCU to the Rx port of the LimeSDR Mini board using the SMA cable.
    \item Connect the Nucleo board to your computer.
    \item Connect the LimeSDR Mini board to your computer using the extension USB cable.
    To avoid any performance loss, do not connect the LimeSDR Mini board directly to your computer, as interference from your computer may corrupt the received signal.
    \item If on VirtualBox, pass both devices to the guest system (\textit{"Devices -> USB -> ..."}). On WSL, see install guidelines how to attach the LimeSDR.
\end{enumerate}

\begin{bclogo}[couleur = gray!20, arrondi = 0.2, logo=\bcinfo]{Passing the LimeSDR Mini to the VM}
    Depending on your computer, you may have difficulties to pass the LimeSDR Mini to the guest system.
    When passing the devices in the VirtualBox menu, use the command \texttt{lsusb} in a terminal to verify if the device has correctly been passed.
    If you are unable to pass the board after a few trials, shutdown the VM, \textbf{close the VirtualBox manager and restart it}, and relaunch the VM.
\end{bclogo}

\subsection{Setting up the MCU}

Open the MCU software project \texttt{mcu} with Eclipse (\textit{"File -> Open projects from file system ..."}).
In the configuration file \texttt{Core/Inc/config.h}, ensure that the following macros are defined as
\begin{itemize}
    \item \texttt{ENABLE\_RADIO 1}: initialize the S2LP radio when booting.
    \item \texttt{RUN\_CONFIG EVAL\_RADIO}: use the radio evaluation mode.
    \item \texttt{ENABLE\_UART 1}: enable the UART.
    \item \texttt{DEBUGP 1}: enable the debug prints.
\end{itemize}

Beside the general configuration file, the parameters of the radio evaluation mode are defined in \texttt{Core/Inc/eval\_radio.h}.
Open this file and try to understand the different parameters.

Afterwards, compile the software and flash the Nucleo board. Open a serial console to \texttt{/dev/ttyACM0} and press the button B1 to start transmitting packets.

\begin{bclogo}[couleur = gray!20, arrondi = 0.2, logo=\bcinfo]{MCU build error}
    If you encounter an error when building the MCU software project, this is probably because you need to update the project settings. To do so, please open the \texttt{ioc} and click \textit{Yes} when prompted to migrate the project to the new version. After, \textbf{do not forget} to re-generate. Finally, you should be able to compile the project. Some errors about \texttt{\_getpid} and \texttt{\_kill} may be left, but you can safely ignore them.
\end{bclogo}

\subsection{Running the GNU Radio application}

The teaching team prepared a GNU Radio application ready to receive packets from the S2LP radio.
Open the application \texttt{gr-fsk/apps/eval\_limesdr.grc}. Instead of reading a stream from a file, the application uses the LimeSDR Mini
to retrieve samples. Try to understand the effects of the following variables in the application:
\begin{itemize}
    \item \texttt{packet\_len}: length of the packet, in bytes.
    \item \texttt{rx\_gain}: gain of the amplification stage in the receiver chain, in decibels. It can be change during operation.
    \item \texttt{detect\_threshold\_entry}: threshold of the preamble detection. You can also change it during operation.\\
    Open the Python file \texttt{gr-fsk/python/preamble\_detection.py} to understand its behavior.
    \item The cut-off frequency and transition width of the low pass filter.
\end{itemize}
If the role of some of these parameters are unclear to you, please ask the assistant for clarifications.
Moreover, to allow the calculation of the SNR and to ease the setting of the threshold, the \textit{Noise estimation on query} can be activated during operation through the GUI during operation. It computes the noise power $K$ times over $N$ samples. It can be used to evaluate the noise before you enable the chain with the \textit{Enable chain} check box, or when you change the receiver gain.

When launching the GNU Radio application, the procedure is the following:
\begin{enumerate}
    \item Launch the GNU Radio application.
    \item Perform a noise estimation by clicking on \textit{"Noise estimation query"}.
    \item Update the detection threshold based on the estimated noise standard deviation.
    \item Enable the chain by checking the box \textit{Enable chain}.
    \item Start transmitting packets with the MCU.
\end{enumerate}

To be noted, there is \textit{TX power} variable that can be changed but it will only be useful later on in the project when you will make BER measurements.

\begin{bclogo}[couleur = gray!20, arrondi = 0.2, logo=\bcinfo]{Help, my virtual machine becomes unstable or freezes!}
    If you observe that your virtual machine is unable to smoothly run the GNU Radio application, you may need to increase the number
    of CPU cores and the memory shared with the VM.
\end{bclogo}

\subsection{Optional: Recording a capture}

When debugging GNU Radio blocks, recording a capture is a very useful technique to run the same experiment again in identical conditions.
An application which performs a recording of the samples acquired by the LimeSDR Mini is provided in \texttt{gr-fsk/apps/record\_capture.grc}.
Try to use this application to record a new capture while the MCU transmits packets, and then demodulate it using \texttt{gr-fsk/apps/decode\_capture.grc}.

\section{Live demodulation over the air}

In the ecomonitoring application, the MCU needs to communicate with the gateway over the air, and not using a cable.
The final part of this hands-on session consists in using the antennas to transmit and receive the packets.

\subsection{Frequency allocation}

Since all wireless transmissions use the same medium (i.e., the air), it is important that each group uses a different carrier frequency to avoid interfering with other groups.
To this end, please use the following frequency allocation scheme, in which each group is allocated a bandwidth of \SI{2}{\mega\hertz}:
\begin{table}[h]
    \centering
    \begin{tabular}{c|c}
        Group number & Allocated carrier frequency\\
        \hline
        Group A & \SI{860}{\mega\hertz}\\
         Group B & \SI{862}{\mega\hertz}\\
         Group C & \SI{864}{\mega\hertz}\\
         ... & ...
    \end{tabular}
    \caption{Frequency allocation scheme among the groups.}
    \label{tab:freq_alloc}
\end{table}

The allocated carrier frequency must be configured both in the MCU and in the GNU Radio Companion.
\begin{itemize}
    \item In the MCU code, open the file \texttt{Core/Inc/s2lp.h} which contains the parameters used by the radio.
    Try to understand the different parameters and modify the macro \texttt{BASE\_FREQ} to your allocated carrier frequency.
    \item In GNU Radio, modify the variable \texttt{carrier\_freq}.
\end{itemize}
For more details on the different parameters and modes in which the S2LP radio can be configured to, have a look at Tables 12 (page 13) and 40 (page 28) in the S2-LP datasheet.

\subsection{Running the GNU Radio application}

Once the carriers are correctly configured, replace the SMA cable with the antennas and run the application.
Since transmitting over the air implies that the signal at the receiver will have a much weaker power compared to a transmission over a cable,
it is necessary to increase the Rx gain in the LimeSDR Mini. This can be done using the GUI when the application is running. \textbf{Run the application and set the gain at \SI{60}{\decibel}.}
You should now be able to receive packets over the air.
You might need to try several distances (between 1 and 5 meters) and different values of the variable \texttt{detect\_threshold} to achieve a functional communication.

\subsection{Measuring the noise level}

When receiving a packet, the receiver estimates its signal-to-noise ratio (SNR), defined as
\begin{equation}
    \textrm{SNR} = \frac{P}{\sigma^2},
\end{equation}
with $P$ being the power of the received signal and $\sigma^2$ is the variance of the additive white gaussian noise (AWGN).
Estimating the SNR is required to experimentally evaluate the performance of the receiver.
The theoretical explanation of the estimation of both random variables has been provided in the appendix of the previous hands-on.
In short, the power of the received signal is dynamically estimated using the preamble of the packet.
On the contrary, the power of the noise is a fixed value which has to be estimated before the application is launched.

In practice, an estimate of the noise power $\sigma^2$ can be obtained using the \textit{"Noise estimation query"} block during operation and \textbf{when the antenna is connected to the LimeSDR Mini}.
Several estimation of $\sigma^2$ are made and their mean is calculated.

\begin{enumerate}
    \item Estimate the noise power using the query with a Rx gain of \SI{60}{\decibel}.
    \item Repeat the operation by bypassing the low-pass filter (right-click on the block to enable/bypass it, or select if and press D or E).
    What is the effect of the low-pass filter on the noise power?
\end{enumerate}
