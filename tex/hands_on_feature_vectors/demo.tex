\section{Demo D5: double buffering acquisition and classification}
%
% During this LELEC2102 project, we expect you to be able to present quick demos on specific topics, as scheduled in the program for this semester. They can be seen as small checkpoints to make sure you master the important basic blocks of the project. These demos should take you less time than writing a detailed report while still providing a good occasion to develop new skills, learn by doing and finally get some feedback along the way! There is no need to write anything for a demo but you must make sure it will run "live" smoothly. \\

The \textbf{demo D5 on Friday November 29} will consist in checking you master the provided code and you are be able to modify it accordingly for some improvements during the second semester. For this demo on Friday, we expect you to:
%
\begin{itemize}
    \item Show an acquisition of a full feature vector from a single button press.
    \item Show a feature vector-long melspectrogram acquired with the jack cable, then with the microphone.
    \item Show a classification result obtained live from acquisition with your MCU.
\end{itemize}
